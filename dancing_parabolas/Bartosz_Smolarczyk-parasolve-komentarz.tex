\documentclass[12pt]{article}

\usepackage{amssymb}
\usepackage{graphicx}
\usepackage{polski}
\usepackage[utf8]{inputenc}
\usepackage{geometry}
\usepackage{amsmath}
\usepackage{amsfonts} 
\allowdisplaybreaks
\geometry{
	a4paper,
	margin=20mm
}

\usepackage{lastpage}
\usepackage{fancyhdr}
\usepackage{mathtools}
\usepackage{bm}
\usepackage[shortlabels]{enumitem}
\usepackage{float}
\usepackage{hyperref}
\usepackage{listings}
\usepackage{color}

\definecolor{dkgreen}{rgb}{0,0.6,0}
\definecolor{gray}{rgb}{0.5,0.5,0.5}
\definecolor{mauve}{rgb}{0.58,0,0.82}

\lstset{frame=tb,
  language=C++,
  aboveskip=3mm,
  belowskip=3mm,
  showstringspaces=false,
  columns=flexible,
  basicstyle={\small\ttfamily},
  numbers=none,
  numberstyle=\tiny\color{gray},
  keywordstyle=\color{blue},
  commentstyle=\color{dkgreen},
  stringstyle=\color{mauve},
  breaklines=true,
  breakatwhitespace=true,
  tabsize=3
}

\pagestyle{fancy}
\fancyhf{}
\renewcommand{\headrulewidth}{0pt}
\cfoot{Strona \thepage\ z \pageref{LastPage}}

\newcommand{\RR}{\mathbb{R}}
\newcommand{\ZZ}{\mathbb{Z}}
\newcommand{\NN}{\mathbb{N}}
\newcommand{\CC}{\mathbb{C}}
\newcommand{\KK}{\mathbb{K}}
\newcommand{\QQ}{\mathbb{Q}}

\newcommand{\0}{x_{n}}
\newcommand{\1}{x_{n+1}}
\newcommand{\2}{x_{n+2}}
\newcommand{\3}{x_{n+3}}

\newcommand\bigforall{\mbox{\Large $\mathsurround0pt\forall$}}
\newcommand\bigexists{\mbox{\Large $\mathsurround0pt\exists$}}

\begin{document}
\begin{center}
	{\Huge Metody numeryczne}\\[7pt]
	Komentarz 3. zadania laboratoryjnego\\
	Bartosz Smolarczyk
\end{center}
\begin{enumerate}[1.]
\large\bf\item Wybór pierwiastka równania kwadratowego \boldsymbol{$\3$} \smallskip \\
\normalfont
\normalsize
Przeprowadzone rozumowanie opieramy na wybranym w następnym punkcie wzorze $(4)$. \medskip \\
Chcielibyśmy wybierać kolejne pierwiastki $\3$ tak, aby nie doprowadzić do sytuacji, w której dzielimy przez 0. Łatwo zauważyć, że może dojść do tego jedynie podczas \mbox{wyliczania} wartości $\mathcal{M} = C \mp \sqrt{\Delta}$ (pierwiastek z ujemnej liczby nie stanowi problemu przy \mbox{założeniu}, że obsługiwane są liczby zespolone). Mając to na uwadze, przyjmujemy \mbox{następującą} \mbox{politykę} wyboru $\3$:
\begin{enumerate}[$\bullet$]
\item Jeżeli $\mathcal{M} \neq 0$ dla obydwu pierwiastków, to wybieramy ten, który wartością jest bliżej $\frac{\0 + \1}{2}$,

\item Jeżeli $\mathcal{M} = 0$ dla jednego z pierwiastków, to wybieramy ten, dla którego $\mathcal{M}  \neq 0$,

\item Jeżeli $\mathcal{M} = 0$ dla obydwu pierwiastków, to losujemy z jednakowym prawdopodobieństwem $\mathcal{M} = -2 \vee \mathcal{M} = 2$. Co prawda wpływa to nieco na ostateczny wygląd ciągu, ale pozwala kontynuować iterację. Zauważmy, że wówczas wzór obliczeniowy upraszcza się do $\3 = \2 \pm f[\2]$.

\end{enumerate}
\bigskip
\large\bf\item Wybór wzoru obliczeniowego wartości \boldsymbol{$\3$} \smallskip \\
\normalfont
\normalsize
Wybieramy wzór obliczeniowy (4):
$$
\3 = \2 - \frac{2f[\2]}{C \mp \sqrt{\Delta}}
$$
Co prawda obliczenie kolejnego wyrazu ciągu wymaga tak samo jak we wzorze (3) wyznaczenia $C$ oraz $\Delta$, przez co jesteśmy zmuszeni obliczać wielokrotnie różnice dzielone i wykonywać przy tym wysoce niepożądane ze względu na niedokładności dzielenie. \mbox{Zauważmy} jednak, że w liczniku (4) występuje $f[\2] = f(\2)$, a nie tak jak w mianowniku (3) $f[\2,\1,\0]$. Pozwala nam to na ograniczenie liczby dzieleń wykonywanych w ramach jednej iteracji. Jako dodatkowy powód możemy przyjąć przeprowadzone testy - wzór ten cechował się mniejszą liczbą potrzebnych do znalezienia miejsca zerowego iteracji.\\

\large\bf\item Obsługa liczb zespolonych \smallskip \\
\normalfont
\normalsize
Zaimplementowana przy pomocy pakietu \texttt{cmath}.\\

\large\bf\item Dowody równoważności \smallskip \\
\normalfont
\normalsize
W poniższych przekształceniach pamiętamy z treści \mbox{zadania}, że zachodzi:
\begin{align*}
C &= f[\2, \1] + f[\2, \0] - f[\1, \0] \\[1ex]
\Delta &= C^2 - 4f[\2]f[\2, \1, \0]
\end{align*}
\begin{enumerate}[$\bullet$]
\pagebreak
\newpage
\item (1) $\Leftrightarrow$ (2) \medskip \\
Za pomocą odpowiednich przekształceń pokażemy, że (1) = (2). Zapiszmy tę równość, od razu redukując składnik $f[\2]$:
\begin{align*}
&f[\2,\1] \cdot (x - \2) + f[\2, \1, \0] \cdot (x - \2)(x - \1) = \\[1ex]
&= C(x - \2) + f[\2, \1, \0] \cdot (x - \2)^2 \\
\Leftrightarrow & \\
&f[\2, \1, \0] \cdot (x - \2)(x - \1) - f[\2, \1, \0] \cdot (x - \2)^2 = \\[1ex]
&= C(x - \2) - f[\2,\1] \cdot (x - \2) \\
\Leftrightarrow & \\
&f[\2, \1, \0] \cdot (x - \2) \cdot \left( x - \1 - x + \2 \right) = \\[1ex]
&= (x - \2) \cdot \left( C - f[\2, \1] \right) \\
\Leftrightarrow & \\
&f[\2, \1, \0] \cdot (x - \2) \cdot (\2 - \1) = \\[1ex]
&= (x - \2) \cdot \left(f[\2, \1] + f[\2, \0] - f[\1, \0] - f[\2, \1] \right) \\
\Leftrightarrow & \\
& (x - \2) \cdot \frac{f[\1, \0] - f[\2, \1]}{\0 - \2} \cdot (\2 - \1) = \\[1ex]
&= (x - \2) \cdot \left(f[\2, \0] - f[\1, \0]\right) \\
\Leftrightarrow & \\
& (x - \2) \cdot \frac{\frac{f[\0] - f[\1]}{\0 - \1} - \frac{f[\1] - f[\2]}{\1 - \2}}{\0 - \2} \cdot (\2 - \1) = \\[1ex]
&= (x - \2) \cdot \left(\frac{f[\0] - f[\2]}{\0 - \2} - \frac{f[\0] - f[\1]}{\0 - \1} \right) \\
\Leftrightarrow & \\
& (x - \2) \cdot \frac{\frac{(f[\0] - f[\1]) \cdot (\1 - \2) - (f[\1] - f[\2]) \cdot (\0 - \1)}{(\0 - \1) \cdot (\1 - \2)}}{\0 - \2} \cdot (\2 - \1) = \\[1ex]
&= (x - \2) \cdot \frac{(f[\0] - f[\2]) \cdot (\0 - \1) - (f[\0] - f[\1]) \cdot (\0 - \2)}{(\0 - \2) \cdot (\0 - \1)} \\
\Leftrightarrow & \\
& \scriptstyle (x - \2) \cdot \frac{(f[\0] - f[\1]) \cdot (\1 - \2) - (f[\1] - f[\2]) \cdot (\0 - \1)}{(\0 - \2) \cdot (\0 - \1) \cdot (\1 - \2)} \cdot (\2 - \1) \; = \\[1ex]
&= \; \scriptstyle (x - \2) \cdot \frac{f[\0] \cdot \0 - f[\0] \cdot \1 - f[\2] \cdot \0 + f[\2] \cdot \1 - f[\0] \cdot \0 + f[\0] \cdot \2 + f[\1] \cdot \0 - f[\1] \cdot \2}{(\0 - \2) \cdot (\0 - \1)} \\
&\text{redukujemy w pierwszym równaniu } (\1 - \2) \text{ z } (\2 - \1) \text{, pamiętając o} \\ 
&\text{pomnożeniu licznika przez } (-1): \\
\Leftrightarrow & \\
& \scriptstyle (x - \2) \cdot \frac{(f[\1] - f[\2]) \cdot (\0 - \1) - (f[\0] - f[\1]) \cdot (\1 - \2)}{(\0 - \2) \cdot (\0 - \1)}\; = \\[1ex]
&= \; \scriptstyle (x - \2) \cdot \frac{f[\0] \cdot \0  - f[\0] \cdot \0 - f[\0] \cdot \1 + f[\0] \cdot \2 + f[\1] \cdot \0 - f[\1] \cdot \2 - f[\2] \cdot \0 + f[\2] \cdot \1}{(\0 - \2) \cdot (\0 - \1)} \\
\Leftrightarrow & \\
& \scriptstyle (x - \2) \cdot \frac{f[\1] \cdot \0 - f[\1] \cdot \1 - f[\2] \cdot \0 + f[\2] \cdot \1 - f[\0] \cdot \1 + f[\0] \cdot \2 + f[\1] \cdot \1 - f[\1] \cdot \2}{(\0 - \2) \cdot (\0 - \1)}\; = \\[1ex]
&= \; \scriptstyle (x - \2) \cdot \frac{- f[\0] \cdot \1 + f[\0] \cdot \2 + f[\1] \cdot \0 - f[\1] \cdot \2 - f[\2] \cdot \0 + f[\2] \cdot \1}{(\0 - \2) \cdot (\0 - \1)} \\
\Leftrightarrow & \\
& \scriptstyle (x - \2) \cdot \frac{ - f[\0] \cdot \1 + f[\0] \cdot \2 + f[\1] \cdot \0 - f[\1] \cdot \1 + f[\1] \cdot \1 - f[\1] \cdot \2 - f[\2] \cdot \0 + f[\2] \cdot \1}{(\0 - \2) \cdot (\0 - \1)}\; = \\[1ex]
&= \; \scriptstyle (x - \2) \cdot \frac{f[\0] \cdot \2 - f[\0] \cdot \1 + f[\1] \cdot \0 - f[\1] \cdot \2 + f[\2] \cdot \1  - f[\2] \cdot \0}{(\0 - \2) \cdot (\0 - \1)} \\
\Leftrightarrow & \\
& \scriptstyle (x - \2) \cdot \frac{ - f[\0] \cdot \1 + f[\0] \cdot \2 + f[\1] \cdot \0 - f[\1] \cdot \2 - f[\2] \cdot \0 + f[\2] \cdot \1}{(\0 - \2) \cdot (\0 - \1)}\; = \\[1ex]
&= \; \scriptstyle (x - \2) \cdot \frac{f[\0] \cdot (\2 - \1) + f[\1] \cdot (\0 - \2) + f[\2] \cdot (\1 - \0)}{(\0 - \2) \cdot (\0 - \1)} \\
\Leftrightarrow & \\
& \scriptstyle (x - \2) \cdot \frac{f[\0] \cdot \2 - f[\0] \cdot \1 + f[\1] \cdot \0 - f[\1] \cdot \2 + f[\2] \cdot \1 - f[\2] \cdot \0}{(\0 - \2) \cdot (\0 - \1)}\; = \\[1ex]
&= \; \scriptstyle (x - \2) \cdot \frac{f[\0] \cdot (\2 - \1) + f[\1] \cdot (\0 - \2) + f[\2] \cdot (\1 - \0)}{(\0 - \2) \cdot (\0 - \1)} \\
\Leftrightarrow & \\
& (x - \2) \cdot \frac{f[\0] \cdot (\2 - \1) + f[\1] \cdot (\0 - \2) + f[\2] \cdot (\1 - \0)}{(\0 - \2) \cdot (\0 - \1)}\; = \\[1ex]
&= (x - \2) \cdot \frac{f[\0] \cdot (\2 - \1) + f[\1] \cdot (\0 - \2) + f[\2] \cdot (\1 - \0)}{(\0 - \2) \cdot (\0 - \1)}
\end{align*}\\
Tym samym pokazaliśmy, że (1) $\Leftrightarrow$ (2). $\quad \square$
\bigskip
\bigskip
\item (3) $\Leftrightarrow$ (4) \medskip \\
Wymnażamy licznik i mianownik ułamka w równości (3) przez $(C \mp \sqrt{\Delta})$ (uwaga - zmiana znaków) i korzystamy z wzoru skróconego mnożenia na różnicę kwadratów:
\begin{align*}
\3 &= \2 - \frac{C \pm \sqrt{\Delta}}{2f[\2, \1, \0]} =\\[1ex]
&= \2 - \frac{C \pm \sqrt{\Delta}}{2f[\2, \1, \0]} \cdot \frac{C \mp \sqrt{\Delta}}{C \mp \sqrt{\Delta}} =\\[1ex]
&= \2 - \frac{C^2 - \Delta}{2f[\2, \1, \0] \cdot (C \mp \sqrt{\Delta})} =\\[1ex]
&= \2 - \frac{C^2 - \left(C^2 - 4f[\2]f[\2, \1, \0]\right)}{2f[\2, \1, \0] \cdot (C \mp \sqrt{\Delta})} =\\[1ex]
&= \2 - \frac{4f[\2]f[\2, \1, \0]}{2f[\2, \1, \0] \cdot (C \mp \sqrt{\Delta})} =\\[1ex]
&= \2 - \frac{2f[\2]}{C \mp \sqrt{\Delta}} =\\[1ex]
&= \3
\end{align*}
Tym samym pokazaliśmy, że (3) $\Leftrightarrow$ (4). $\quad \square$ \\\\\\
\end{enumerate}
\end{enumerate}
\begin{figure}[h]
\centering
\href{https://www.youtube.com/watch?v=AGA_1HsP_C0&ab_channel=damaziom1}{\includegraphics[scale=0.25]{parabole.png}}
\end{figure}
\end{document}

