\documentclass[12pt]{article}

\usepackage{amssymb}
\usepackage{graphicx}
\usepackage{polski}
\usepackage[utf8]{inputenc}
\usepackage{geometry}
\usepackage{amsmath}
\usepackage{amsfonts} 
\allowdisplaybreaks
\geometry{
	a4paper,
	margin=20mm
}

\usepackage{lastpage}
\usepackage{fancyhdr}
\usepackage{mathtools}
\pagestyle{fancy} 

\fancyhf{}
\renewcommand{\headrulewidth}{0pt}
\cfoot{Strona \thepage\ z \pageref{LastPage}}

\newcommand{\RR}{\mathbb{R}}
\newcommand{\ZZ}{\mathbb{Z}}
\newcommand{\NN}{\mathbb{N}}
\newcommand{\CC}{\mathbb{C}}
\newcommand{\KK}{\mathbb{K}}
\newcommand{\QQ}{\mathbb{Q}}
\newcommand{\code}{\texttt}

\newcommand\bigforall{\mbox{\Large $\mathsurround0pt\forall$}}
\newcommand\bigexists{\mbox{\Large $\mathsurround0pt\exists$}}

\begin{document}
\begin{center}
	{\Huge Metody numeryczne}\\[7pt]
	Komentarz 1. zadania laboratoryjnego\\
	Bartosz Smolarczyk
\end{center}
\begin{center}
\textbf{Poprawność implementacji i złożoność obliczeniowa}
\end{center}
\begin{itemize}
\item \code{\_\_init\_\_(self, a:np.ndarray, b:np.ndarray, c:np.ndarray) -> dTridiag}
\vspace{1mm}\\
Metoda zapisuje w tworzonym obiekcie klasy \code{dTridiag} wektory \code{a}, \code{b} oraz \code{c} jako \code{diagonal}, \code{hyperdiagonal} i \code{subdiagonal}, wykonując jednocześnie konwersję ich zawartości do typu \code{float}. Dodatkowo zapisywane są rozmiar diagonalii (\code{main\_len} = $n$), sub- \mbox{i hiperdiagonalii} (\code{side\_len}) oraz odległość wierszowa/kolumnowa \code{d} między główną diagonalą a subdiagonalą/hiperdiagonalą.\\
Na koniec metoda wywołuje \code{\_\_LU\_decompose()}, która wyznacza rozkład LU (bez \mbox{wyboru} elementu dominującego, obliczamy go od razu aby być gotowym na wielokrotne wywołania \code{dTridiag.solve}). Rozkład jest zapisywany jako wektory \code{L\_subdiag}, \code{U\_diag} oraz \code{U\_hyperdiag} ponieważ w macierzach L oraz U niezerowe są jedynie główne diagonale oraz odpowiednio sub- i hiperdiagonala (nie poświęcamy pamięci na diagonalę macierzy L złożoną z samych 1).\\
\\
\textbf{Złożoność obliczeniowa:} O$(n)$ - liniowe zapisywanie wektorów diagonalii, liniowe obliczanie rozkładu LU.\\
\textbf{Złożoność pamięciowa:} O$(n)$ - wszystkie dane są trzymane wewnątrz kilku wektorów rozmiaru $n$.
\vspace{5mm}
\item \code{dTridiag.dot(self, v: np.ndarray) -> np.ndarray}
\vspace{1mm}\\
Metoda wykonuje konwersję wektora \code{v} do typu \code{float}. Następnie najpierw mnoży \mbox{wartości} diagonali przez wartości wektora \code{v}, a na koniec do iloczynu dodaje wyniki mnożeń sub- \mbox{i hiperdiagonali} przez odpowiednie fragmenty wektora \code{v}.\\
\\
\textbf{Złożoność czasowa:} O$(n)$ - liniowe mnożenie elementów wektorów i ich dodawanie.\\
\textbf{Złożoność pamięciowa:} O$(n)$ - dodatkowa pamięć na wektor wynikowy rozmiaru $n$.
\vspace{5mm}
\item \code{dTridiag.solve(self, y: np.ndarray) -> np.ndarray}
\vspace{1mm}\\
Metoda wykonuje konwersję wektora \code{y} do typu \code{float}. Następnie rozwiązuje układy \mbox{równań} $Lz = y$ oraz $Ux = z$, ostatecznie zwracając szukany wektor $x$. Obliczenia \mbox{wykonuje} blokowo, rozwiązując $d$ równań na raz (są to proste równania z jedną niewiadomą), później podstawia wyznaczone wartości do kolejnego bloku $d$ równań z dwiema niewiadomymi sprowadzając je do równań z jedną niewiadomą i tak do rozwiązania całego układu.\\
\\
\textbf{Złożoność czasowa:} O$(n)$ - dla każdego wiersza $L$ oraz $U$ wykonujemy pojedyncze operacje zmiennoprzecinkowe.\\
\textbf{Złożoność pamięciowa:} O$(n)$ - dodatkowa pamięć na wektor wynikowy rozmiaru $n$.
\end{itemize}
\end{document}

